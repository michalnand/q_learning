
\begin{titlepage}
\phantom.

\bigskip

\begin{center}
{\sc\LARGE ŽILINSKÁ UNIVERZITA V ŽILINE}
\medskip

{\sc\Large Fakulta riadenia a informatiky}

\vfill\vfill\vfill\vfill

{\sc\LARGE AUTOREFERÁT DIZERTAČNEJ PRÁCE}

\medskip

{\large Študijný odbor: {\bf Aplikovaná informatika}}
\end{center}


\vfill\vfill\vfill\vfill

Žilina, Apríl, 2016 \\
\\
autor : Ing. Michal Chovanec \\
vedúci: prof. Ing. Juraj Miček, PhD

\medskip



\end{titlepage}








\begin{titlepage}
\phantom.

\bigskip

\begin{center}
{\sc\Large ŽILINSKÁ UNIVERZITA V ŽILINE}
\medskip

{\sc\Large Fakulta riadenia a informatiky}



\vfill\vfill\vfill\vfill
{\bf Ing. Michal Chovanec}
\vfill\vfill\vfill\vfill

Autoreferát dizertačnej práce

Aproximácia funkcie ohodnotení v algoritmoch Q-learning neurónovou sieťou



 \vfill\vfill\vfill\vfill



na získanie akademického titulu  philosophiae doctor (v skratke PhD.) \\
v študijnom programe doktorandského štúdia \\
aplikovaná informatika \\
v študijnom odbore: \\
9.2.9 aplikovaná informatika
\end{center}
\end{titlepage}





\begin{titlepage}


Dizertačná práca bola vypracovaná v dennej forme doktorandského štúdia forme doktorandského štúdia
na katedre technickej kybernetiky, Fakulte riadenia a informatiky Žilinskej univerzity v Žiline


\vfill

Predkladateľ: \\
Ing. Michal Chovanec \\
Katedra technickej kybernetiky \\
Fakulta riadenia a informatiky \\
Žilinská univerzita v Žiline \\

\vfill

Školiteľ:	\\
prof. Ing. Juraj Miček, PhD \\
Katedra technickej kybernetiky \\
Fakulta riadenia a informatiky \\
Žilinská univerzita v Žiline \\

\vfill
\vfill
\vfill


Oponenti:
\vfill
Titul, meno a priezvisko  : \\
Názov pracoviska          : \\
\vfill
Titul, meno a priezvisko  : \\
Názov pracoviska          : \\
\vfill
Titul, meno a priezvisko  : \\
Názov pracoviska          : \\

\vfill
\vfill

Autoreferát bol rozoslaný dňa: .............................................

\vfill

Obhajoba dizertačnej práce sa koná dňa ....................................... o ............ h.
pred komisiou pre obhajobu dizertačnej práce schválenou odborovou komisiou v študijnom
odbore 9.2.9 aplikovaná informatika, v študijnom programe aplikovaná informatika,
vymenovanou dekanom Fakulty riadenia a informatiky Žilinskej univerzity v Žiline.


\vfill


prof. Ing.Martin Klimo, PhD.
predseda odborovej komisie
študijného programu aplikovaná informatika
v študijnom odbore 9.2.9 aplikovaná informatika
Fakulta riadenia a informatiky
Žilinská univerzita
Univerzitná 8215/1
010 26 Žilina


\end{titlepage}















%--------------------------------------------------------------------------------------
%%% slovensky abstrakt

\begin{abstract}

\noindent
{\sc Michal Chovanec:} {\em   Aproximácia funkcie ohodnotení v algoritmoch Q-learning neurónovou sieťou}

\noindent
Žilinská Univerzita v~Žiline,
Fakulta riadenia a informatiky,
Katedra technickej kybernetiky.

\noindent
Vedúci: prof. Ing. Juraj Miček, PhD

\noindent
FRI ŽU v~Žiline, 2016

\bigskip

Práca sa zaoberá aproximáciou funkcie ohodnotení konania agenta, v algoritmoch Q-learning.
V priestoroch s malým počtom stavov predstavuje vhodné riešenie tabuľka.
Pre prípady veľkého počtu stavov je tabuľkové riešenie ťažko vypočítateľné. Je tak nutné použiť
aproximáciu. Vhodným kandidátom je neurónová sieť. Tradičné riešenie doprednej
siete je však nepoužiteľné z dôvodov nemožnosti takúto sieť učiť. V práci
je preto venovaný priestor neurónovej sieti bázických funkcií ktorú už je možné
na daný problém trénovať iteračnými metódami.

\end{abstract}
